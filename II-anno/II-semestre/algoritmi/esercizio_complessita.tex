\documentclass[11pt,a4paper,twoside]{report}
% draft mode prevents link generation

\usepackage[italian]{babel}
% per date e ToC in italiano
\usepackage{hyperref}
\usepackage{nameref}
\usepackage[table]{xcolor}
\usepackage{amsmath}        % matematica
\usepackage{amsthm}         % teoremi
\usepackage{amssymb}        % simboli
\usepackage{amsfonts}       % font matematicosi
\usepackage{mathrsfs}
\usepackage{mathtools}
\usepackage{thmbox}
\usepackage{centernot}      % per semplificare
\usepackage{pgfplots}       % per i grafici
\pgfplotsset{compat=1.5}    % consigliata compatibilita'
                            % con la versione 1.5

\usepackage{fullpage}       % troppo margine normalmente
\usepackage{parskip}        % preferisco spazio fra i paragrafi
                            % all'indentazione sulla prima riga
\usepackage{fancyhdr}
\usepackage[makeroom]{cancel}

% algorithms
\usepackage{algorithm}
\usepackage{algpseudocode}

\floatname{algorithm}{Algoritmo}
\renewcommand{\algorithmicrequire}{\textbf{Input:}}
\renewcommand{\algorithmicensure}{\textbf{Output:}}

\theoremstyle{plain}
\newtheorem{axiom}{Assioma}
\newtheorem{theorem}{Teorema}[section]
\newtheorem{lem}[theorem]{Lemma}
\newtheorem{prop}[theorem]{Proposizione}
\newtheorem{cor}[theorem]{Corollario}

\theoremstyle{definition}
\newtheorem{defn}{Definizione}[section]
\newtheorem{esercizio}{Esercizio}

\theoremstyle{remark}
\newtheorem{remark}{Remark}
\newtheorem{oss}{Osservazione}
\newtheorem{caso}{Caso}
\newtheorem{fact}{Fatto}
\newtheorem{claim}{Claim}

\newenvironment{exmp}[1][]
{ \par\medskip\textbf{Esempio\ #1\unskip\,:}\\*[1pt]\rule{\textwidth}{0.4pt}\\*[1pt] }
{ \\*[1pt]\rule{\textwidth}{0.4pt}\medskip }

\newenvironment{smallpmatrix}
{\left( \begin{smallmatrix}}
{\end{smallmatrix} \right)}

\newenvironment{completepmatrix}[1]
{\left(\begin{array}{*{#1}{c}|c}}
{\end{array}\right)}

\pagestyle{fancyplain}
\fancyhead{} % clear all header fields
\fancyfoot{} % clear all footer fields
\fancyfoot[C]{\today}
\fancyfoot[LE,RO]{\thepage}
\renewcommand{\headrulewidth}{0pt}
\renewcommand{\footrulewidth}{0pt}

\usetikzlibrary{shapes,arrows,calc,fit,backgrounds,positioning,automata}

\newcommand{\parity}[1]{\mathcal{Par}\left( { #1 } \right)}
% Languages & Automata
\newcommand{\langs}[1]{\mathcal{L}\left( {#1} \right)}
\newcommand{\lang}[1]{L\left( {#1} \right)}
\newcommand{\langor}{\ | \ }
\newcommand{\kleenestar}{\star}
\newcommand{\emptystring}{\varepsilon}
\newcommand{\stackend}{\$}
\renewcommand{\leadsto}{\vdash}

\newcommand{\bigo}[1]{O\left( {#1} \right)}
\newcommand{\probp}{\code{P}}
\newcommand{\probnp}{\code{NP}}
\newcommand{\probnpc}{\code{NP-C}}
\newcommand{\probexp}{\code{EXP}}
\newcommand{\covers}{<\!\cdot}      % simbolo di 'copre'
\newcommand{\scalar}{\centerdot}
\newcommand{\compl}{\mathsf{c}}     % C complementare
\newcommand{\dotcup}{\mathaccent\cdot\cup}  % unione disgiunta
\newcommand{\lateralsx}[1]{{}_{#1}\!\sim}   % equivalenza sinistra
\newcommand{\lateraldx}[1]{\sim_{#1}}       % equivalenza destra
\newcommand{\mcm}{\operatorname{mcm}}       % mcm
\newcommand{\mcd}{\operatorname{MCD}}       % MCD
\newcommand{\pow}[1]{<\!{#1}\!>}            % gruppo delle potenze (<a>)
\newcommand{\abs}[1]{\left\lvert{#1}\right\rvert}   % valore assoluto
\newcommand{\intsup}[1]{\left\lceil{#1}\right\rceil}   % intero superiore
\newcommand{\intinf}[1]{\left\lfloor{#1}\right\rfloor}   % intero inferiore
\newcommand{\norm}[1]{\left\lVert{#1}\right\rVert}  % norma di un vettore
\newcommand{\divides}{\bigm|}       % simbolo di 'divide'
\newcommand{\definition}{\overset{\underset{\mathrm{def}}{}}{=}}    % uguale con 'definizione' sopra
\newcommand{\supop}{\vee}           % simbolo dell'operatore sup
\newcommand{\infop}{\wedge}         % simbolo dell'operatore inf
\newcommand{\subgroupset}{\mathscr{S}}      % insieme dei sottogruppi (S corsiva)
\newcommand{\solutions}{\mathscr{S}}
\newcommand{\parts}[1]{\mathbb{P} \left( {#1} \right)}             % insieme delle parti
\newcommand{\naturals}{\mathbb{N}}          % insieme dei naturali
\newcommand{\integers}{\mathbb{Z}}          % insieme degli interi
\newcommand{\rationals}{\mathbb{Q}}         % insieme dei razionali
\newcommand{\reals}{\mathbb{R}}             % insieme dei reali
\newcommand{\complexes}{\mathbb{C}}         % insieme dei complessi
\newcommand{\field}{\mathbb{K}}             % simbolo di campo generico
\newcommand{\subfield}{\mathbb{F}}          % simbolo di sottocampo generico
\newcommand{\matrices}{\mathfrak{M}}        % insieme delle matrici
\newcommand{\nullelement}{\underline{0}}    % elemento neutro
\newcommand{\var}{\operatorname{Var}}       % varianza 
\newcommand{\seq}[3]{{#1}_{#2},\ldots,\,{#1}_{#3}}
\newcommand{\prob}[2][P]{#1 \left( {#2} \right)}
\newcommand{\probcond}[3][P]{#1 \left( #2 | #3 \right)}
\newcommand{\expect}[1]{E \left( #1 \right)}
\newcommand{\mass}[2][X]{p_{#1} \left( #2 \right)}
\newcommand{\distr}[2][X]{F_{#1} \left( #2 \right)}
\newcommand{\image}[1]{Im_{#1}}
\newcommand{\cov}{\operatorname{Cov}}
\newcommand{\detname}{\operatorname{det}}
\newcommand{\sgn}{\operatorname{sgn}}
\renewcommand{\det}[1]{\detname \left( {#1} \right)}
\newcommand{\algcompl}[2][A]{{\mathcal{#1}}_{#2}}
\newcommand{\agg}[1]{\operatorname{Agg} \left( {#1} \right)}
\newcommand{\id}{id}
\newcommand{\sfrac}[2]{{#1}/{#2}}
\newcommand{\requiv}[1][]{\ \varepsilon_{#1} \ }
\newcommand{\abslog}[1]{\log \left( \abs{#1} \right)}
\newcommand{\substitute}[2]{\left. {#1} \; \right|_{#2}}
\newcommand{\increment}[1]{{\left. {#1} \; \right|}}
\newcommand{\adj}{\sim}
\newcommand{\code}[1]{\texttt{#1}}
\newcommand{\isa}{\code{is-a}}
% intervalli aperti e chiusi
\newcommand{\ooint}[2]{\left( {#1}, {#2} \right)}
\newcommand{\coint}[2]{\left[ {#1}, {#2} \right)}
\newcommand{\ocint}[2]{\left( {#1}, {#2} \right]}
\newcommand{\ccint}[2]{\left[ {#1}, {#2} \right]}
\newcommand{\stereotipe}[1]{\ll \code{#1} \gg}

\renewcommand{\labelitemi}{$-$}
\let\oldforall\forall
\renewcommand{\forall}{\ \oldforall \ }
\let\oldexists\exists
\renewcommand{\exists}{\ \oldexists \ }
\renewcommand{\implies}{\Rightarrow}
\renewcommand{\iff}{\Leftrightarrow}

\makeatletter
\let\orgdescriptionlabel\descriptionlabel\renewcommand*{\descriptionlabel}[1]{%
\let\orglabel\label
\let\label\@gobble
\phantomsection
\edef\@currentlabel{#1}%
%\edef\@currentlabelname{#1}%
\let\label\orglabel
\orgdescriptionlabel{#1}%
}
\makeatother

\makeatletter
\providecommand*{\diff}%
    {\@ifnextchar^{\DIfF}{\DIfF^{}}}
\def\DIfF^#1{%
    \mathop{\mathrm{\mathstrut d}}%
        \nolimits^{#1}\gobblespace}
\def\gobblespace{%
        \futurelet\diffarg\opspace}
\def\opspace{%
    \let\DiffSpace\!%
    \ifx\diffarg(%
        \let\DiffSpace\relax
    \else
        \ifx\diffarg[%
            \let\DiffSpace\relax
        \else
            \ifx\diffarg\{%
                \let\DiffSpace\relax
            \fi\fi\fi\DiffSpace}

\providecommand*{\deriv}[1][]{%
    \frac{\diff}{\diff #1}}
%\providecommand*{\deriv}[2][]{%
%    \frac{\diff^{#1}}{\diff #2^{#1}}}
\providecommand*{\fderiv}[3][]{%
    \frac{\diff^{#1}#2}{\diff #3^{#1}}}
\providecommand*{\pderiv}[3][]{%
    \frac{\partial^{#1}#2}%
        {\partial #3^{#1}}}

\newcommand{\dx}[1][x]{\, \diff {#1}}

\author{Michele Laurenti \\ \href{mailto:asmeikal@me.com}{asmeikal@me.com}}
\date{\today}

\usepackage{algorithm}
\usepackage{algpseudocode}

\floatname{algorithm}{Algoritmo}
% \renewcommand{\algorithmicrequire}{\textbf{precondizioni:}}
% \renewcommand{\algorithmicensure}{\textbf{Output:}}

\begin{document}
\setcounter{chapter}{1}
\setcounter{section}{1}

\begin{esercizio}
Si dimostri che il problema del ricoprimento di archi tramite vertici in un grafo determina un linguaggio \code{NP}-completo. Si utilizzi per la dimostrazione una riduzione dal linguaggio \code{NP}-completo delle formule booleane soddisfacibili. Si può assumere che ogni formula è espressa in forma normale congiuntiva e che ogni clausola contiene esattamente tre letterali.
\end{esercizio}

Il problema del sottografo completo \`e in \code{NP}: il ``certificato'' verificabile in tempo polinomiale \`e il vettore caratteristico (di cardinalit\`a $k$) che rappresenta un sottoinsieme di $G$. In tempo polinomiale si pu\`o controllare che ogni vertice nel vettore caratteristico sia adiacente a tutti gli altri vertici appartenenti al vettore caratteristico.

Dobbiamo dimostrare che vale:
\[
\text{3-soddisfacibilit\`a } \le_P \text{ sottografo completo}
\]

La funzione di riduzione, per ogni formula booleana $\varphi$ in terza forma normale congiuntiva, costruisce un grafo $G(V,E)$ con $\abs{V} = 3 \, n$ associando ad ogni nodo un letterale in una delle $n$ clausole di $\varphi$. Inseriamo un arco fra due nodi se i due nodi corrispondono a letterali di clausole diverse che non si contraddicono. Ossia, dati due letterali $l_i^j$ e $l_h^k$ appartenenti alle clausole $j$ e $k$, vale che $j \neq k$ e che $l_i^j \neq \bar{l_h^k}$.

Dire che la formula \`e soddisfacibile vuol dire affermare che per ogni clausola esiste un letterale vero, che vuol dire affermare che per ogni clausola esiste un letterale che non contraddice gli altri. I nodi associati a questi letterali formano un sottografo completo. A un'istanza $<G,k>$ del problema corrisponde una formula con $k$ clausole.
 
Il problema del vertex cover \`e in \code{NP}: il ``certificato'' verificabile in tempo polinomiale \`e il vettore caratteristico (di cardinalit\`a $k$) che rappresenta un sottoinsieme di $G$. In tempo polinomiale si possono rimuovere tutti i vertici contenuti nel vettore caratteristico da $G$. Se dopo questa operazione $G$ ha ancora almeno un arco, il vettore caratteristico non \`e un vertex cover.

Vogliamo dimostrare che vale questo:
\[
\text{sottografo completo } \le_P \text{ vertex cover}
\]

Il problema ``esiste un sottografo completo del grafo $G$ di cardinalit\`a $k$'' pu\`o essere ricondotto al problema del vertex cover.

Sia $L$ il linguaggio:
\[
L = \{ <G,k> : \text{ esiste un sottografo completo del grafo $G$ di cardinalit\`a $k$}\}
\]
e sia $L'$ il linguaggio:
\[
L' = \{ <G,k> : \text{ esiste un vertex cover del grafo $G$ di cardinalit\`a $k$}\}
\]
Affermiamo questo:
\begin{equation}
\label{equiv_vertex_cover_grafo_completo}
<G,k> \in L \iff <\bar{G}, n - k> \in L'
\end{equation}
La riduzione polinomiale dal linguaggio dei sottografi completi al linguaggio dei vertex cover \`e la funzione che mappa la coppia $<G,k>$ nella coppia $<\bar{G}, n - k>$. Sappiamo che questa funzione \`e polinomiale.

Resta da dimostrare l'equivalenza \ref{equiv_vertex_cover_grafo_completo}.

\begin{proof}[di \ref{equiv_vertex_cover_grafo_completo}]
Parte ``se''.

Sia $S$ un sottografo completo di $G$, con $\abs{S} = k$, e sia $S' = G \setminus S$. Chiaramente $\abs{S'} = n - k$. $S'$ \`e un vertex cover di $\bar{G}$.

Per ogni arco $(x,y) \in \bar{G}$, almeno uno fra $x$ e $y$ non appartiene a $S$ (perch\'e $S$ \`e completo). Quindi, almeno uno fra $x$ e $y$ appartiene a $S'$, e quindi $(x,y)$ \`e coperto in $S'$.

Parte ``solo se''.
 
Sia $S'$ un vertex cover di $\bar{G}$, con $\abs{S'} = n - k$, e sia $S = \bar{G} \setminus S'$. Chiaramente $\abs{S} = k$. $S$ \`e un sottografo completo di $G$.

Per ogni arco $(x,y) \in \bar{G}$, almeno uno fra $x$ e $y$ sono in $S'$. Se n\'e $x$ n\'e $y$ sono in $S'$, allora $(x,y) \notin \bar{G}$, e quindi $(x,y) \in G$. Poich\'e $x,y \notin S' \iff x,y \in S$, se $(x,y) \in G$ allora $x,y \in S$, e quindi $S$ \`e un sottografo completo di cardinalit\`a $k$.
\end{proof}

\end{document}
