\documentclass[11pt,a4paper,twoside]{report}
% draft mode prevents link generation

\usepackage[italian]{babel}
% per date e ToC in italiano
\usepackage{hyperref}
\usepackage{nameref}
\usepackage[table]{xcolor}
\usepackage{amsmath}        % matematica
\usepackage{amsthm}         % teoremi
\usepackage{amssymb}        % simboli
\usepackage{amsfonts}       % font matematicosi
\usepackage{mathrsfs}
\usepackage{mathtools}
\usepackage{thmbox}
\usepackage{centernot}      % per semplificare
\usepackage{pgfplots}       % per i grafici
\pgfplotsset{compat=1.5}    % consigliata compatibilita'
                            % con la versione 1.5

\usepackage{fullpage}       % troppo margine normalmente
\usepackage{parskip}        % preferisco spazio fra i paragrafi
                            % all'indentazione sulla prima riga
\usepackage{fancyhdr}
\usepackage[makeroom]{cancel}

% algorithms
\usepackage{algorithm}
\usepackage{algpseudocode}

\floatname{algorithm}{Algoritmo}
\renewcommand{\algorithmicrequire}{\textbf{Input:}}
\renewcommand{\algorithmicensure}{\textbf{Output:}}

\theoremstyle{plain}
\newtheorem{axiom}{Assioma}
\newtheorem{theorem}{Teorema}[section]
\newtheorem{lem}[theorem]{Lemma}
\newtheorem{prop}[theorem]{Proposizione}
\newtheorem{cor}[theorem]{Corollario}

\theoremstyle{definition}
\newtheorem{defn}{Definizione}[section]
\newtheorem{esercizio}{Esercizio}

\theoremstyle{remark}
\newtheorem{remark}{Remark}
\newtheorem{oss}{Osservazione}
\newtheorem{caso}{Caso}
\newtheorem{fact}{Fatto}
\newtheorem{claim}{Claim}

\newenvironment{exmp}[1][]
{ \par\medskip\textbf{Esempio\ #1\unskip\,:}\\*[1pt]\rule{\textwidth}{0.4pt}\\*[1pt] }
{ \\*[1pt]\rule{\textwidth}{0.4pt}\medskip }

\newenvironment{smallpmatrix}
{\left( \begin{smallmatrix}}
{\end{smallmatrix} \right)}

\newenvironment{completepmatrix}[1]
{\left(\begin{array}{*{#1}{c}|c}}
{\end{array}\right)}

\pagestyle{fancyplain}
\fancyhead{} % clear all header fields
\fancyfoot{} % clear all footer fields
\fancyfoot[C]{\today}
\fancyfoot[LE,RO]{\thepage}
\renewcommand{\headrulewidth}{0pt}
\renewcommand{\footrulewidth}{0pt}

\usetikzlibrary{shapes,arrows,calc,fit,backgrounds,positioning,automata}

\newcommand{\parity}[1]{\mathcal{Par}\left( { #1 } \right)}
% Languages & Automata
\newcommand{\langs}[1]{\mathcal{L}\left( {#1} \right)}
\newcommand{\lang}[1]{L\left( {#1} \right)}
\newcommand{\langor}{\ | \ }
\newcommand{\kleenestar}{\star}
\newcommand{\emptystring}{\varepsilon}
\newcommand{\stackend}{\$}
\renewcommand{\leadsto}{\vdash}

\newcommand{\bigo}[1]{O\left( {#1} \right)}
\newcommand{\probp}{\code{P}}
\newcommand{\probnp}{\code{NP}}
\newcommand{\probnpc}{\code{NP-C}}
\newcommand{\probexp}{\code{EXP}}
\newcommand{\covers}{<\!\cdot}      % simbolo di 'copre'
\newcommand{\scalar}{\centerdot}
\newcommand{\compl}{\mathsf{c}}     % C complementare
\newcommand{\dotcup}{\mathaccent\cdot\cup}  % unione disgiunta
\newcommand{\lateralsx}[1]{{}_{#1}\!\sim}   % equivalenza sinistra
\newcommand{\lateraldx}[1]{\sim_{#1}}       % equivalenza destra
\newcommand{\mcm}{\operatorname{mcm}}       % mcm
\newcommand{\mcd}{\operatorname{MCD}}       % MCD
\newcommand{\pow}[1]{<\!{#1}\!>}            % gruppo delle potenze (<a>)
\newcommand{\abs}[1]{\left\lvert{#1}\right\rvert}   % valore assoluto
\newcommand{\intsup}[1]{\left\lceil{#1}\right\rceil}   % intero superiore
\newcommand{\intinf}[1]{\left\lfloor{#1}\right\rfloor}   % intero inferiore
\newcommand{\norm}[1]{\left\lVert{#1}\right\rVert}  % norma di un vettore
\newcommand{\divides}{\bigm|}       % simbolo di 'divide'
\newcommand{\definition}{\overset{\underset{\mathrm{def}}{}}{=}}    % uguale con 'definizione' sopra
\newcommand{\supop}{\vee}           % simbolo dell'operatore sup
\newcommand{\infop}{\wedge}         % simbolo dell'operatore inf
\newcommand{\subgroupset}{\mathscr{S}}      % insieme dei sottogruppi (S corsiva)
\newcommand{\solutions}{\mathscr{S}}
\newcommand{\parts}[1]{\mathbb{P} \left( {#1} \right)}             % insieme delle parti
\newcommand{\naturals}{\mathbb{N}}          % insieme dei naturali
\newcommand{\integers}{\mathbb{Z}}          % insieme degli interi
\newcommand{\rationals}{\mathbb{Q}}         % insieme dei razionali
\newcommand{\reals}{\mathbb{R}}             % insieme dei reali
\newcommand{\complexes}{\mathbb{C}}         % insieme dei complessi
\newcommand{\field}{\mathbb{K}}             % simbolo di campo generico
\newcommand{\subfield}{\mathbb{F}}          % simbolo di sottocampo generico
\newcommand{\matrices}{\mathfrak{M}}        % insieme delle matrici
\newcommand{\nullelement}{\underline{0}}    % elemento neutro
\newcommand{\var}{\operatorname{Var}}       % varianza 
\newcommand{\seq}[3]{{#1}_{#2},\ldots,\,{#1}_{#3}}
\newcommand{\prob}[2][P]{#1 \left( {#2} \right)}
\newcommand{\probcond}[3][P]{#1 \left( #2 | #3 \right)}
\newcommand{\expect}[1]{E \left( #1 \right)}
\newcommand{\mass}[2][X]{p_{#1} \left( #2 \right)}
\newcommand{\distr}[2][X]{F_{#1} \left( #2 \right)}
\newcommand{\image}[1]{Im_{#1}}
\newcommand{\cov}{\operatorname{Cov}}
\newcommand{\detname}{\operatorname{det}}
\newcommand{\sgn}{\operatorname{sgn}}
\renewcommand{\det}[1]{\detname \left( {#1} \right)}
\newcommand{\algcompl}[2][A]{{\mathcal{#1}}_{#2}}
\newcommand{\agg}[1]{\operatorname{Agg} \left( {#1} \right)}
\newcommand{\id}{id}
\newcommand{\sfrac}[2]{{#1}/{#2}}
\newcommand{\requiv}[1][]{\ \varepsilon_{#1} \ }
\newcommand{\abslog}[1]{\log \left( \abs{#1} \right)}
\newcommand{\substitute}[2]{\left. {#1} \; \right|_{#2}}
\newcommand{\increment}[1]{{\left. {#1} \; \right|}}
\newcommand{\adj}{\sim}
\newcommand{\code}[1]{\texttt{#1}}
\newcommand{\isa}{\code{is-a}}
% intervalli aperti e chiusi
\newcommand{\ooint}[2]{\left( {#1}, {#2} \right)}
\newcommand{\coint}[2]{\left[ {#1}, {#2} \right)}
\newcommand{\ocint}[2]{\left( {#1}, {#2} \right]}
\newcommand{\ccint}[2]{\left[ {#1}, {#2} \right]}
\newcommand{\stereotipe}[1]{\ll \code{#1} \gg}

\renewcommand{\labelitemi}{$-$}
\let\oldforall\forall
\renewcommand{\forall}{\ \oldforall \ }
\let\oldexists\exists
\renewcommand{\exists}{\ \oldexists \ }
\renewcommand{\implies}{\Rightarrow}
\renewcommand{\iff}{\Leftrightarrow}

\makeatletter
\let\orgdescriptionlabel\descriptionlabel\renewcommand*{\descriptionlabel}[1]{%
\let\orglabel\label
\let\label\@gobble
\phantomsection
\edef\@currentlabel{#1}%
%\edef\@currentlabelname{#1}%
\let\label\orglabel
\orgdescriptionlabel{#1}%
}
\makeatother

\makeatletter
\providecommand*{\diff}%
    {\@ifnextchar^{\DIfF}{\DIfF^{}}}
\def\DIfF^#1{%
    \mathop{\mathrm{\mathstrut d}}%
        \nolimits^{#1}\gobblespace}
\def\gobblespace{%
        \futurelet\diffarg\opspace}
\def\opspace{%
    \let\DiffSpace\!%
    \ifx\diffarg(%
        \let\DiffSpace\relax
    \else
        \ifx\diffarg[%
            \let\DiffSpace\relax
        \else
            \ifx\diffarg\{%
                \let\DiffSpace\relax
            \fi\fi\fi\DiffSpace}

\providecommand*{\deriv}[1][]{%
    \frac{\diff}{\diff #1}}
%\providecommand*{\deriv}[2][]{%
%    \frac{\diff^{#1}}{\diff #2^{#1}}}
\providecommand*{\fderiv}[3][]{%
    \frac{\diff^{#1}#2}{\diff #3^{#1}}}
\providecommand*{\pderiv}[3][]{%
    \frac{\partial^{#1}#2}%
        {\partial #3^{#1}}}

\newcommand{\dx}[1][x]{\, \diff {#1}}

\author{Michele Laurenti \\ \href{mailto:asmeikal@me.com}{asmeikal@me.com}}
\date{\today}

\usepackage{algorithm}
\usepackage{algpseudocode}

\floatname{algorithm}{Algoritmo}
\renewcommand{\algorithmicrequire}{\textbf{precondizioni:}}
% \renewcommand{\algorithmicensure}{\textbf{Output:}}

\begin{document}

\setcounter{chapter}{1}

\section{MST e SPT}

\begin{esercizio}
Sia $G$ un grafo non diretto, connesso e pesato dove i pesi sono tutti diversi tra loro. Sia $T$ un minimo albero di copertura di $G$, sia $s$ un nodo e $T_s$ l'albero dei cammini minimi da $s$ verso tutti gli altri nodi di $G$. Dimostrare oppure fornire un controesempio che $T_s$ e $T$ condividono almeno un arco.
\end{esercizio}

Sia $(s,v) \in T_s$ un arco tale che $p(s,v) < p(s,x) \forall x \adj s$.

Supponiamo $(s,v) \notin T$. Sia $T' = T \cup \{(s,v)\}$. $T'$ contiene necessariamente un ciclo.

Sia $(s,w)$ un arco appartenente al ciclo. $(s,w)$ \`e tale che $p(s,w) > p(s,v)$, quindi $T'' = T' \setminus \{ (s,w) \}$ \`e un albero di copertura, e $p(T'') < P(T)$. Assurdo, quindi $(s,v)$ appartiene necessariamente a $T$.

\clearpage

\section{Salto doppio}

\begin{esercizio}
In un vettore $V$ di $n$ interi si dice salto doppio un indice $i$, $1 \le i < n$, tale che $V[i + 1] - V[i] \ge 2$.
\begin{enumerate}
    \item Provare che un vettore $V$ di $n \ge 2$ interi tale che $V[n]-V[1] \ge n$, ha almeno un salto doppio.
    \item Descrivere un algoritmo che, dato un vettore $V$ di $n \ge 2$ interi tale che $V[n] - V[1] \ge n$, trova un salto doppio in tempo $O\left(\log (n)\right)$.
\end{enumerate}
\end{esercizio}

Primo punto: se non c'\`e un salto doppio fra gli elementi $V[i]$ e $V[i+1]$, allora vale:
\[
V[i+1] - V[i] \le 1 \implies V[i+1] \le V[i] + 1
\]
Quindi, se non c'\`e un salto doppio nel vettore $V$, deve valere che:
\[
V[n] \le V[n-1] + 1 \le \ldots \le V[n - i] + i \le V[n - (n - 1)] + n - 1 = V[1] + n - 1
\]
Quindi, deve valere che $V[n] \le V[1] + n - 1 \implies V[n] - V[1] \le n - 1$, in contraddizione con l'ipotesi che $V[n] - V[1] \ge n$.

Secondo punto: dimostriamo che un vettore $A$ contiene un salto doppio fra gli indici $a < b$ se $A[b] - A[a] > b - a$.

\begin{proof}
Poich\'e c'\`e un salto doppio fra due elementi consecutivi se $A[i+1] > A[i] + 1$, vale che:
\[
A[b] > A[b-1] + 1 > \ldots > A[b - (b - a)] + b - a = A[a] + b - a
\]
\end{proof}

\begin{algorithm}
\begin{algorithmic}
\Require $A[b] - A[a] > b - a$
\Function{Salto\_Doppio}{$A$: array, $a,b$: indici}
    \If{$b = a + 1$}
        \State \Return $a$
    \Else
        \State $m \gets \frac{a + b}{2}$
        \If{$A[m] - A[a] > m - a$}
            \State \Return \Call{Salto\_Doppio}{$A, a, m$}
        \Else
            \State \Return \Call{Salto\_Doppio}{$A, m, b$}
        \EndIf
    \EndIf
\EndFunction
\end{algorithmic}
\caption{\label{alg_salto_doppio}trovare la posizione di un salto doppio in un array ordinato di interi}
\end{algorithm}

La complessit\`a dell'algoritmo \ref{alg_salto_doppio} \`e uguale alla complessit\`a della ricerca binaria, ossia $O\left(\log (n)\right)$.

\clearpage

\section{Coppie}

\begin{esercizio}
Abbiamo una sequenza $S = (s_1,s_2,\ldots,s_n)$ di interi positivi. Una sottosequenza $S'$ di $S$ si dice valida se per ogni coppia di elementi consecutivi di $S$ almeno un elemento della coppia compare in $S'$. Il valore di una sottosequenza valida \`e la somma dei suoi elementi. Ad esempio, se $S = (1,2,3,5,4,6,7)$ allora $S' = (1,3,6)$ non \`e valida, $S' = (2,5,4,7)$ \`e valida di valore 18 e $S'' = (2,3,4,6)$ \`e valida di valore 15.
\begin{enumerate}
    \item Descrivere un algoritmo che, data la sequenza $S$, calcola il valore minimo di una sottosequenza valida in tempo $O(n)$.
    \item Descrivere poi un algoritmo che trova una sottosequenza valida di valore minimo.
\end{enumerate}
\end{esercizio}

Consideriamo una tabella $M$. Per ogni valore $i = 1 \ldots n$, $M[i]$ \`e il valore minimo della somma di una sottosequenza valida contenente l'elemento $S[i]$.

I casi di base sono, banalmente:
\begin{enumerate}
    \item $M[1] = S[1]$
    \item $M[2] = S[2]$
    \item $M[3] = S[3] + \min(S[2], S[1])$
\end{enumerate}
Il caso generico \`e, per ogni $i > 3$:
\[
M[i] = S[i] + \min (M[i-1], M[i-2], M[i-3])
\]
L'elemento $i$-esimo deve essere nella sottosequenza, e bisogna prendere poi uno dei tre elementi precedenti (per ottenere un valore minimo), o non tutte le coppie consecutive sarebbero coperte dalla sottosequenza.
\begin{center}
\begin{tabular}{*{5}{l|}l}
\hline
$\cdots$ & $i-3$ & $i-2$ & $i-1$ & $i$ & $\cdots$ \\
\hline 
\end{tabular}
\end{center}

\begin{proof}
La correttezza dei casi base \`e banalmente dimostrata.

Supponiamo l'algoritmo sia corretto per un certo $i$, ossia per ogni $j \le i$, $M[j]$ \`e il valore minimo della somma di una sottosequenza valida contenente l'elemento $S[j]$. $M[i+1]$ \`e banalmente il valore minimo della somma di una sottosequenza valida contenente l'elemento $S[i+1]$, essendo pari a:
\[
M[i+1] = S[i+1] + \min (M[i], M[i-1], M[i-2])
\]
Non si pu\`o scegliere un indice minore di $i-2$. Se si scegliesse, ad esempio, l'indice $i-3$, la sequenza non sarebbe pi\`u valida, poich\'e l'elemento $i-1$ non sarebbe coperto.
\end{proof}

$M[n-1]$ e $M[n]$ sono il valore minimo della somma di una sottosequenza valida che copre tutto l'array. Il minore fra i due \`e il valore che cerchiamo.

\begin{algorithm}
\begin{algorithmic}
\Function{Coppie}{$S$: array, $n$: intero}
    \State $M \gets$ tabella di dimensione $n$
    \State $M[1] \gets S[1]$
    \State $M[2] \gets S[2]$
    \State $M[3] \gets S[3] + \min(S[2], S[1])$
    \For{$i \gets 4 \ldots n$}
        \State $M[i] \gets S[i] + \min (M[i-1], M[i-2], M[i-3])$
    \EndFor
    \State \Return $\min(M[n], M[n-1])$
\EndFunction
\end{algorithmic}
\caption{\label{alg_coppie}trovare il valore minimo di una sottosequenza valida}
\end{algorithm}

La complessit\`a dell'algoritmo \ref{alg_coppie} per trovare il valore minimo di una sottosequenza valida \`e $O(n)$. Il caso per $n \le 3$ \`e banale, e assumiamo quindi valga $n > 3$.

L'algoritmo \ref{alg_sol_coppie} per trovare una soluzione a partire dalla tabella calcolata dall'algoritmo \ref{alg_coppie} ha anch'esso complessit\`a $O(n)$.

\begin{algorithm}
\begin{algorithmic}
\Function{Sol\_Coppie}{$S$: array, $M$: tabella, $n$: intero}
    \State $sol \gets$ array booleano di $n$ elementi, inizializzato a 0
    \State $v \gets \min (M[n], M[n-1])$
    \If{$v = M[n]$}
        \State $i \gets n$
    \Else
        \State $i \gets n - 1$
    \EndIf
    \State $sol[i] \gets 1$
    \State $v \gets v - S[i]$
    \While{$v > 0$}
        \If{$v = M[i-1]$}
            \State $i \gets i - 1$
        \ElsIf{$v = M[i-2]$}
            \State $i \gets i - 2$
        \Else
            \State $i \gets i - 3$
        \EndIf
        \State $v \gets v - S[i]$
        \State $sol[i] \gets 1$
    \EndWhile
    \State \Return $sol$
\EndFunction
\end{algorithmic}
\caption{\label{alg_sol_coppie} trovare una sottosequenza valida}
\end{algorithm}

\clearpage

\section{Punto fisso}

\begin{esercizio}
In un vettore $V$ di $n$ interi si dice punto fisso un indice $i$, $1 \le i \le n$, tale che $V[i] = i$. Dato un vettore ordinato $V$ di $n$ interi distinti vogliamo determinare se $V$ contiene punti fissi.
\begin{enumerate}
    \item Dare lo pseudo-codice di un algoritmo che nel caso particolare in cui gli interi del vettore sono tutti interi positivi risolva il problema in $O(1)$ tempo.
    \item Dare lo pseudo-codice di un algoritmo che nel caso generale risolva il problema in tempo $O \left(\log (n) \right)$.
\end{enumerate}
\end{esercizio}

Se gli interi del vettore sono ordinati, distinti, e tutti positivi, $V$ contiene un punto fisso $\iff V[0] = 0$.

Supponiamo per assurdo esista un punto fisso di indice $i$ nonostante $V[0] > 0 \implies V[0] \ge 1$. Poich\'e il vettore \`e composto da interi distinti e ordinati, $\forall i < n$ vale che $V[i+1] \ge V[i] + 1$. Quindi:
\[
V[i] \ge V[i-1] + 1 \ge \ldots \ge V[1] + i - 1 \ge V[0] + i \ge i + 1
\]
Assurdo.

Se gli interi sono ordinati, distinti, e non tutti positivi, vale che esiste un punto fisso con indice maggiore o uguale a $i$ se $V[i] \le i$. Sfruttando questa propriet\`a possiamo applicare la ricerca binaria per trovare un punto fisso, se c'\`e.

\begin{algorithm}
\begin{algorithmic}
\Require $V[a] \le a$
\Function{Punti\_Fissi}{$V$: vettore, $a,b$: indici}
    \If{$a = b$}
        \If{$V[a] = a$}
            \State \Return $a$
        \Else
            \State \Return $-1$
        \EndIf
    \Else
        \State $m \gets \frac{a + b}{2}$
        \If{$V[m] > m$}
            \State \Return \Call{Punti\_Fissi}{$V,a,m$}
        \Else
            \State \Return \Call{Punti\_Fissi}{$V,m,b$}
        \EndIf
    \EndIf
\EndFunction
\end{algorithmic}
\caption{\label{alg_punti_fissi}trovare un punto fisso}
\end{algorithm}

La complessit\`a dell'algoritmo \ref{alg_punti_fissi} \`e la complessit\`a della ricerca binaria, quindi $O \left(\log (n) \right)$.

\clearpage

\section{Sottosequenze}

\begin{esercizio}
Data una sequenza $S$ di interi positivi si vuole trovare una sottosequenza di $S$ senza elementi in posizioni consecutive di somma massima. Ad esempio, per $S = (1,5,4,6,10,3,2,9)$ una soluzione \`e $(-,5,-,-,10,-,-,9)$ e vale 24 (anche $(1,-,4,-,10,-,-,9)$ \`e una soluzione), mentre $(1,-,-,6,10,-,-,9)$ non \`e una sottosequenza ammissibile perché contiene gli elementi 6 e 10 in posizioni consecutive.
\begin{enumerate}
    \item Dare lo pseudo-codice di un algoritmo che risolve il problema in $O(n)$.
    \item Dare poi lo pseudo-codice di un algoritmo che trova una sottosequenza ottimale.
\end{enumerate}
\end{esercizio}

La lunghezza della sottosequenza deve essere almeno $\frac{n}{2}$, poich\'e gli interi nella sequenza $S$ sono tutti positivi, e per ogni sottosequenza $S'$ di lunghezza minore di $\frac{n}{2}$ esiste almeno un intero $x \in S$ non adiacente a nessun intero nella sequenza $S'$, e tale per cui il valore di $S' \cup \{ x \}$ \`e maggiore o uguale al valore di $S'$.

Consideriamo una tabella $M$. Per ogni valore $i = 1 \ldots n$, $M[i]$ \`e il valore massimo della somma di una sottosequenza senza elementi in posizioni consecutive contenente l'elemento $i$-esimo.

I casi base sono:
\begin{enumerate}
    \item $M[1] = S[1]$
    \item $M[2] = S[2]$
    \item $M[3] = S[3] + S[1]$
\end{enumerate}
Il caso generico \`e, per $i > 3$:
\[
M[i] = S[i] + \max ( M[i-2], M[i-3] )
\]
La soluzione sar\`a il $\max(M[n], M[n-1])$.

\begin{proof}
La correttezza dei casi base \`e dimostrabile banalmente.

Supponiamo l'algoritmo sia corretto per ogni $j \le i$, ossia $M[j]$ \`e il valore massimo della somma di una sottosequenza senza elementi in posizioni consecutive contenente l'elemento $j$-esimo. $M[i+1]$ contiene l'elemento $S[i+1]$, e la maggiore fra le sequenze che terminano in $i-1$ o $i-2$. Quindi $M[i+1]$ \`e il valore massimo della somma di una sottosequenza senza elementi in posizioni consecutive, e contiene l'elementi $(i+1)$-esimo.
\end{proof}

\begin{algorithm}
\begin{algorithmic}
\Function{Sottosequenze}{$S$: array di interi, $n$: intero}
    \State $M \gets$ tabella di lunghezza $n$
    \State $M[1] \gets S[1]$
    \State $M[2] \gets S[2]$
    \State $M[3] \gets S[3] + S[1]$
    \For{$i \gets 4 \ldots n$}
        \State $M[i] \gets S[i] + \max(M[i-2], M[i-3])$
    \EndFor
    \State \Return $\max (M[n], M[n-1])$
\EndFunction
\end{algorithmic}
\caption{\label{alg_sottosequenze}calcolare il valore di una sottosequenza di somma massima}
\end{algorithm}

La complessit\`a dell'algoritmo \ref{alg_sottosequenze} \`e $O(n)$.  Il caso per $n \le 3$ \`e banale, e assumiamo quindi valga $n > 3$.

\begin{algorithm}
\begin{algorithmic}
\Function{Sol\_Sottosequenze}{$S$: array di interi, $n$: intero, $M$: tabella}
    \State $Sol \gets$ array booleano di lunghezza $n$ inizializzato a 0
    \State $v \gets \max (M[n], M[n-1])$
    \If{$M[n] = v$}
        \State $i \gets n$
    \Else
        \State $i \gets n - 1$
    \EndIf
    \State $Sol[i] \gets 1$
    \State $v \gets v - S[i]$
    \While{$v > 0$}
        \If{$v = M[i-2]$}
            \State $i \gets i - 2$
        \Else
            \State $i \gets i - 3$
        \EndIf
        \State $Sol[i] \gets 1$
        \State $v \gets v - S[i]$
    \EndWhile
    \State \Return $Sol$
\EndFunction
\end{algorithmic}
\caption{\label{alg_sol_sottosequenze}calcolare una sottosequenza di somma massima}
\end{algorithm}

La complessit\`a dell'algoritmo \ref{alg_sol_sottosequenze} per trovare una sottosequenza di somma massima a partire dalla tabella calcolata dall'algoritmo \ref{alg_sottosequenze} \`e $O(n)$.

\clearpage

\section{Zaino rivisitato}

\begin{esercizio}
Nella versione classica abbiamo uno Zaino di capacit\`a $C$ ed n oggetti ciascuno con un suo valore $v_i$ ed un suo peso $p_i$, $1 \le i \le n$. Possiamo mettere nello zaino un qualunque sottoinsieme degli oggetti il cui peso complessivo non superi la capacit\`a $C$ e vogliamo trovare il sottoinsieme di valore massimo. Nella versione rivisitata disponiamo di un numero arbitrario di copie di ciascuno degli $n$ oggetti e quindi pi\`u copie di uno stesso oggetto possono essere inserite nello zaino allo scopo di ottenere una soluzione di valore massimo. Ad esempio per uno zaino di capacit\`a $C = 18$ e tre oggetti con valore e peso specificati dalle seguenti tabelle:
\begin{center}
\begin{tabular}{lll}
\multicolumn{3}{c}{Peso} \\
\hline
$p_1$ & $p_2$ & $p_3$ \\
9 & 5 & 4
\end{tabular}
\qquad
\begin{tabular}{lll}
\multicolumn{3}{c}{Valore} \\
\hline
$v_1$ & $v_2$ & $v_3$ \\
6 & 4 & 3
\end{tabular}
\end{center}
la combinazione di oggetti che massimizza il valore dello zaino rispettandone la capacit\`a \`e quella che prende due copie del secondo oggetto e due copie del terzo (questa combinazione ha peso 18 e valore 14).
\begin{enumerate}
    \item Dare lo pseudo-codice di un algoritmo che calcola il valore della soluzione ottima in $O(n \cdot C)$.
    \item Dare poi lo pseudo-codice di un algoritmo che produce una soluzione ottimale restituendo in output il numero di copie di ciascun oggetto da inserire nello zaino (per l'esempio l'algoritmo restituir\`a il vettore $(0,2,2)$).
\end{enumerate}
\end{esercizio}

Consideriamo una tabella $T$. Per ogni valore $i = 0 \ldots n$ e $j = 0 \ldots C$, $T[i,j]$ \`e il valore massimo che si pu\`o trasportare in uno zaino di capacit\`a $j$ prendendo pi\`u copie dei primi $i$ oggetti.

I casi base sono $T[0,j] = 0$ e $T[i,0] = 0$, per $i = 0 \ldots n$ e $j = 0 \ldots C$.

Il caso generico \`e:
\[
T[i,j] = 
\begin{cases}
T[i-1,j] & \text{ se } p[i] > j \\
\max \left( T[i-1,j], v[i] + T \left[ i, j-p[i] \right] \right) & \text{ altrimenti}
\end{cases}
\]
per $i = 0 \ldots n$ e $j = 0 \ldots C$.

Il primo caso corrisponde al caso in cui l'oggetto $i$ ha peso maggiore di $j$ (la capacit\`a dello zaino). 

Il secondo caso corrisponde al caso in cui il peso dell'oggetto \`e minore della capacit\`a dello zaino: possiamo quindi scegliere o meno una copia dell'oggetto $i$.

\begin{algorithm}
\begin{algorithmic}
\Function{Zaino\_Bis}{$p$: array pesi, $v$: array valori, $n,C$: interi}
    \State $T \gets$ tabella di dimensione $(n+1) \times (C+1)$
    \For{$i \gets 0 \ldots n$}
        \State $T[i,0] \gets 0$
    \EndFor
    \For{$j \gets 1 \ldots C$}
        \State $T[0,j] \gets 0$
    \EndFor
    \For{$i \gets 1 \ldots n$}
        \For{$j \gets 1 \ldots C$}
            \State $T[i,j] \gets T[i-1,j]$
            \If{$p[i] \le j$}
                \State $T[i,j] \gets \max \left( T[i,j], v[i] + T \left[ i, j-p[i] \right] \right)$
            \EndIf
        \EndFor
    \EndFor
    \State \Return $T[n,C]$
\EndFunction
\end{algorithmic}
\caption{\label{alg_zaino_bis}trovare il valore di una soluzione ottima}
\end{algorithm}

La complessit\`a dell'algoritmo \ref{alg_zaino_bis} \`e $O(n \cdot C)$.

\begin{algorithm}
\begin{algorithmic}
\Function{Sol\_Zaino\_Bis}{$T$: tabella, $n, C$: interi, $p$: array pesi}
    \State $sol \gets$ array di $n$ interi
    \State $i \gets n$
    \State $j \gets C$
    \While{$i > 0 \land j > 0$}
        \If{$T[i,j] = T[i-1,j]$}
            \State $i \gets i - 1$
        \Else
            \State $sol[i] \gets sol[i] + 1$
            \State $j \gets j - p[i]$
        \EndIf
    \EndWhile
    \State \Return $sol$
\EndFunction
\end{algorithmic}
\caption{\label{alg_sol_zaino_bis}calcolare una soluzione ottima}
\end{algorithm}

La complessit\`a dell'algoritmo \ref{alg_sol_zaino_bis} per calcolare una soluzione ottima a partire dalla tabella calcolata dall'algoritmo \ref{alg_zaino_bis} \`e $O(n + C)$.

\clearpage

\section{Somma}

\begin{esercizio}
Data una sequenza di $n$ interi positivi $X$ e un intero positivo $s$ vogliamo trovare la pi\`u lunga sottosequenza di $X$ di somma $s$. Ad esempio se $X = (5,2,2,6,1,7,3,5,11,3,6)$ e $s = 25$, la pi\`u lunga sottosequenza \`e lunga 7 $(5,2,2,-,1,7,3,5,-,-,-)$; se $X = (3,3,5,13,3,5)$ e $s = 28$, non ci sono sottosequenze di somma 28.
\begin{enumerate}
    \item Dare lo pseudo-codice di un algoritmo che dati $X$ e $s$ ritorna la lunghezza massima di una sottosequenza di somma $s$ di $X$ in tempo $O(n \cdot s)$ (se non ci sono sottosequenze di somma $s$, ritorna 0).
    \item Dare poi lo pseudo-codice di un algoritmo che ritorna una sottosequenza di lunghezza massima per $X$ e $s$.
\end{enumerate}
\end{esercizio}

Consideriamo una tabella $T$. Per ogni valore $i = 0 \ldots n$ e $j = 0 \ldots s$, $T[i,j]$ \`e la lunghezza massima della sottosequenza dei primi $i$ numeri che hanno somma $j$. Se $T[i,j]$ \`e 0, non esiste nessuna sottosequenza.

I casi base sono $T[0,j] = 0$ e $T[i,0] = 0$, per $i = 0 \ldots n$ e $j = 0 \ldots c$.

Il caso generico \`e:
\[
T[i,j] = 
\begin{cases}
\max (T[i-1,j], 1) & \text{ se } j = X[i] \\
\max \left( T[i-1,j], T \left[ i-1,j-X[i] \right] + 1 \right) & \text{ se } T \left[ i-1,j-X[i] \right] \neq 0 \text{ e } X[i] > j \\
T[i-1,j] & \text{ altrimenti}
\end{cases}
\]
per $i = 0 \ldots n$ e $j = 0 \ldots c$.

Nel primo caso, $j$ \`e esattamente $X[i]$, quindi esiste sicuramente una sottosequenza contenente solo l'elemento $i$ la cui somma \`e $j$. 

Nel secondo caso, $j$ \`e maggiore di $X[i]$, quindi se esiste una sottosequenza lunga $k$ (con $k > 0$) la cui somma \`e $j - X[i]$, esiste una sottosequenza lunga almeno $k + 1$ la cui somma \`e $j$, che contiene l'elemento $i$.

Infine, in generale, il numero di sottosequenze di somma $j$ contenenti i primi $i$ elementi \`e almeno il numero di sottosequenze di somma $j$ contenenti i primi $i-1$ elementi.

\begin{algorithm}
\begin{algorithmic}
\Function{Somma}{$X$: array di interi, $s,n$: interi}
    \State $T \gets$ una tabella $(n+1) \times (s+1)$
    \For{$i \gets 0 \ldots n$}
        \State $T[i,0] \gets 0$
    \EndFor
    \For{$j \gets 1 \ldots s$}
        \State $T[0,j] \gets 0$
    \EndFor
    \For{$i \gets 1 \ldots n$}
        \For{$j \gets 1 \ldots s$}
            \State $T[i,j] \gets T[i-1, j]$
            \If{$(T \left[ i-1,j-X[i] \right] \neq 0 \land X[i] > j) \lor X[i] = j$}
                \State $T[i,j] \gets \max \left( T \left[ i-1,j-X[i] \right] + 1, T[i,j] \right)$
            \EndIf
        \EndFor
    \EndFor
    \State \Return $T[n,s]$
\EndFunction
\end{algorithmic}
\caption{\label{alg_somma}trovare la lunghezza massima di una sottosequenza di somma $s$}
\end{algorithm}

La complessit\`a dell'algoritmo \ref{alg_somma} \`e $O(n \cdot s)$.

\begin{algorithm}
\begin{algorithmic}
\Function{Sol\_Somma}{$T$: tabella, $X$: array interi, $s,n$: interi}
    \State $sol \gets$ array booleano di $n$ elementi, inizializzato a 0
    \While{$n > 0 \land s > 0$}
        \If{$T[n,s] = T[n-1,s]$}
            \State $n \gets n - 1$
        \Else
            \State $sol[n] \gets 1$
            \State $s \gets s - X[n]$
        \EndIf
    \EndWhile
    \State \Return $sol$
\EndFunction
\end{algorithmic}
\caption{\label{alg_sol_somma}trovare una sottosequenza di lunghezza massima e somma $s$}
\end{algorithm}

La complessit\`a dell'algoritmo \ref{alg_sol_somma} che calcola una sottosequenza di lunghezza massima a partire dalla tabella trovata dall'algoritmo \ref{alg_somma} \`e $O(n + s)$.

\begin{table}
\begin{tabular}{r|*{25}{c}}
   &  0 &  1 &  2 &  3 &  4 &  5 &  6 &  7 &  8 &  9 & 10 & 11 & 12 & 13 & 14 & 15 & 16 & 17 & 18 & 19 \\ \hline
 0 &  0 &  0 &  0 &  0 &  0 &  0 &  0 &  0 &  0 &  0 &  0 &  0 &  0 &  0 &  0 &  0 &  0 &  0 &  0 &  0 \\
 1 &  0 &  0 &  0 &  0 &  0 &  1 &  0 &  0 &  0 &  0 &  0 &  0 &  0 &  0 &  0 &  0 &  0 &  0 &  0 &  0 \\
 2 &  0 &  0 &  1 &  0 &  0 &  1 &  0 &  2 &  0 &  0 &  0 &  0 &  0 &  0 &  0 &  0 &  0 &  0 &  0 &  0 \\
 3 &  0 &  0 &  1 &  0 &  2 &  1 &  0 &  2 &  0 &  3 &  0 &  0 &  0 &  0 &  0 &  0 &  0 &  0 &  0 &  0 \\
 4 &  0 &  0 &  1 &  0 &  2 &  1 &  1 &  2 &  2 &  3 &  3 &  2 &  0 &  3 &  0 &  4 &  0 &  0 &  0 &  0 \\
 5 &  0 &  1 &  1 &  2 &  2 &  3 &  2 &  2 &  3 &  3 &  4 &  4 &  3 &  3 &  0 &  4 &  5 &  0 &  0 &  0 \\
 6 &  0 &  1 &  1 &  2 &  2 &  3 &  2 &  2 &  3 &  3 &  4 &  4 &  4 &  3 &  3 &  4 &  5 &  5 &  5 &  4 \\
 7 &  0 &  1 &  1 &  2 &  2 &  3 &  3 &  3 &  4 &  3 &  4 &  4 &  4 &  5 &  5 &  5 &  5 &  5 &  5 &  6 \\
 8 &  0 &  1 &  1 &  2 &  2 &  3 &  3 &  3 &  4 &  3 &  4 &  4 &  4 &  5 &  5 &  5 &  5 &  5 &  6 &  6 \\
 9 &  0 &  1 &  1 &  2 &  2 &  3 &  3 &  3 &  4 &  3 &  4 &  4 &  4 &  5 &  5 &  5 &  5 &  5 &  6 &  6 \\
10 &  0 &  1 &  1 &  2 &  2 &  3 &  3 &  3 &  4 &  4 &  4 &  5 &  4 &  5 &  5 &  5 &  6 &  6 &  6 &  6 \\
11 &  0 &  1 &  1 &  2 &  2 &  3 &  3 &  3 &  4 &  4 &  4 &  5 &  4 &  5 &  5 &  5 &  6 &  6 &  6 &  6 
\end{tabular}
\caption{Tabella costruita dall'algoritmo per $X = (5,2,2,6,1,7,3,5,11,3,6)$ e $s = 19$.}
\end{table}

\clearpage

\section{Supersequenza}

\begin{esercizio}
Date due sequenze $X = (x_1,\ldots,x_n)$ e $Y = (y_1,\ldots,y_m)$ una supersequenza di $X$ e $Y$ \`e una qualsiasi sequenza $Z$ tale che sia $X$ che $Y$ sono sottosequenze di $Z$. Ad esempio, per le sequenze di lettere alberi e libri le seguenti sono supersequenze: \emph{alberilibri}, \emph{albelibri}, \emph{lialberi}, \emph{aliberi}.
\begin{enumerate}
    \item Dare lo pseudo-codice di un algoritmo che date due sequenze $X$ e $Y$, di lunghezze $n$ e $m$, calcola la lunghezza minima di una supersequenza di $X$ e $Y$ in $O(n \cdot m)$.
    \item Dare poi lo pseudo-codice di un algoritmo che ritorna una supersequenza di lunghezza minima di $X$ e $Y$.
\end{enumerate}
\end{esercizio}

Una supersequenza deve contenere gli elementi della sottosequenza comune pi\`u lunga di $X$ e $Y$, e gli elementi di $X$ e $Y$ che non appartengono a questa sottosequenza. Quindi il numero di elementi che deve contenere \`e almeno:
\[
LCS(X,Y) + \abs{X} - LCS(X,Y) + \abs{Y} - LCS(X,Y) = 
\abs{X} + \abs{Y} - LCS(X,Y)
\]
L'algoritmo per calcolare la lunghezza dela sottosequenza pi\`u lunga pu\`o essere usato per calcolare la lunghezza minima di una supersequenza.

\begin{algorithm}
\begin{algorithmic}
\Function{Supersequenza}{$X,Y$: sequenze, $n,m$: interi}
    \State $L \gets$ tabella $(n+1) \times (m+1)$
    \For{$i \gets 0 \ldots n$}
        \State $L[i,0] \gets 0$
    \EndFor
    \For{$j \gets 1 \ldots m$}
        \State $L[0,j] \gets 0$
    \EndFor
    \For{$i \gets 1 \ldots n$}
        \For{$j \gets 1 \ldots m$}
            \If{$X[i] = Y[j]$}
                \State $L[i,j] \gets L[i-1, j-1] + 1$
            \Else
                \State $L[i,j] \gets \max \left( L[i-1,j], L[i,j-1] \right)$
            \EndIf
        \EndFor
    \EndFor
    \State \Return $n + m - L[n,m]$
\EndFunction
\end{algorithmic}
\caption{\label{alg_supersequenza}trovare la lunghezza minima di una supersequenza}
\end{algorithm}
La complessit\`a dell'algoritmo \ref{alg_supersequenza} \`e $O(n \cdot m)$.

Consideriamo due coppie di indici $i_{k}, i_{k+1}, j_{k}, j_{k+1}$, tali che:
\[
X[i_{k}] = Y[j_{k}], \, X[i_{k+1}] = Y[j_{k+1}]
\]
Per ogni $i_k \le i \le i_{k+1}$, e per ogni $j_k \le j \le j_{k+1}$, vale che $X[i] \neq Y[j]$ (esclusi i casi agli estremi). La supersequenza minima $Z$ dovr\`a contenere ogni $X[i]$ e ogni $Y[j]$, e $X$ e $Y$ dovranno essere sue sottosequenze. Quindi, $Z$ dovr\`a contenere una sequenza del tipo:
\[
X[i_{k}], X[i_{k}+1], \ldots, X[i_{k+1}-1], Y[j_{k}+1], \ldots, Y[j_{k+1}-1], X[i_{k+1}]
\]
Le posizioni relative degli elementi di $X$ e di $Y$ che non appartengono a entrambe le stringhe devono essere rispettate, perch\'e $Z$ sia una supersequenza di entrambe le sequenze.

La supersequenza $Z$ deve quindi contenere, fra due elementi $Z[i]$ e $Z[j]$ tali che $Z[i] = X[i_{k}] = Y[j_{k}]$ e $Z[j] = X[i_{k+1}] = Y[j_{k+1}]$, tutti gli elementi $X[i]$ per $i_k \le i \le i_{k+1}$, e tutti gli elementi $Y[j]$ tali che $j_k \le j \le j_{k+1}$.

Ripercorrendo la tabella calcolata dall'algoritmo \ref{alg_supersequenza}, \`e sufficiente aggiungere a $Z$ ogni elemento in comune fra le due sequenze, seguito dagli elementi appartenenti solo a $X$ e solo a $Y$ fino al successivo elemento in comune.

\begin{algorithm}
\begin{algorithmic}
\Function{Sol\_Supersequenza}{$X,Y$: sequenze, $n,m$: interi, $L$: tabella}
    \State $c \gets n + m - L[n,m]$
    \State $sol \gets$ sequenza vuota lunga $c$
    \State $i \gets n$
    \State $j \gets m$
    \While{$c > 0$}
        \If{$X[i] = X[j]$}
            \State $sol[c] \gets X[i]$
            \State $i \gets i - 1$
            \State $j \gets j - 1$
        \ElsIf{$L[i,j] = L[i-1, j]$}
            \State $sol[c] \gets X[i]$
            \State $i \gets i - 1$
        \Else
            \State $sol[c] \gets Y[j]$
            \State $j \gets j - 1$
        \EndIf
        \State $c \gets c - 1$
    \EndWhile
    \State \Return $sol$
\EndFunction
\end{algorithmic}
\caption{\label{alg_sol_supersequenza}trovare una supersequenza di lunghezza minima}
\end{algorithm}

La complessit\`a dell'algoritmo \ref{alg_sol_supersequenza} \`e $O(n + m)$.

\clearpage

\section{Majority element}

\begin{esercizio}
Sia $A[1,\ldots,n]$ un array di $n$ elementi. Un \emph{majority element} di $A$ \`e un elemento che occorre in pi\`u di $\frac{n}{2}$ posizioni di $A$ (quindi se $n = 6$ o $n = 7$, un \emph{majority element} deve occorrere almeno 4  volte in $A$). Assumiamo che gli elementi di $A$ non possono essere ordinati, ma si pu\`o calcolare se due elementi sono uguali (per esempio, gli elementi di $A$ potrebbero essere stringhe). Scrivere lo pseudocodice di un algoritmo $O \left(n \cdot \log (n) \right)$ che prende in input un array $A$ e restituisce ``YES'' se esiste un \emph{majority element} in $A$.
\end{esercizio}

L'algoritmo, per avere la complessit\`a richiesta, deve essere un algoritmo \emph{Divide et Impera}. La funzione di ricorrenza dovr\`a essere:
\[
T(n) = a \, T \left( \frac{n}{a} \right) + \Theta(n)
\]
Un'istanza del problema, ossia un array di $n$ elementi, dovr\`a quindi essere spezzato in istanze dello stesso problema, pi\`u piccole. 

I casi base sono quelli in cui si ha un array di 1 o 2 elementi. Nel primo caso, la risposta \`e ``YES'', nel secondo caso la risposta \`e ``YES'' se i due elementi sono uguali.

Consideriamo il caso composto da pi\`u istanze pi\`u piccole. Per semplicit\`a, consideriamo $a = 2$, e quindi consideriamo due istanze pi\`u piccole. Se almeno uno dei due sottoproblemi risponde ``YES'', l'istanza risponde ``YES'' solo se il numero di elementi uguali al \emph{majority element} di una delle due istanze \`e maggiore della met\`a della lunghezza dell'istanza. Questo controllo si pu\`o fare in tempo lineare.

La funzione usata dall'algoritmo ritorner\`a $-1$ se la sottosequenza presa in considerazione non ha un \emph{majority element}, e l'indice del \emph{majority element} altrimenti.

\begin{algorithm}
\begin{algorithmic}
\Function{Majority\_Element}{$A$: array, $a,b$: interi}
    \If{$b \le a + 1$}
        \If{$A[a] = A[b]$}
            \State \Return $a$
        \Else
            \State \Return $-1$
        \EndIf
    \Else
        \State $m \gets \intinf{\frac{a+b}{2}}$
        \State $l \gets$ \Call{Majority\_Element}{$A,a,m$}
        \State $u \gets$ \Call{Majority\_Element}{$A,m+1,b$}
        \If{$u > 0$}
            \State $c \gets 0$
            \For{$i \gets a \ldots b$}
                \If{$A[i] = A[u]$}
                    \State $c \gets c + 1$
                \EndIf
            \EndFor
            \If{$c > \intinf{\frac{b-a}{2}}$}
                \State \Return $u$
            \EndIf
        \EndIf
        \If{$l > 0$}
            \State $c \gets 0$
            \For{$i \gets a \ldots b$}
                \If{$A[i] = A[l]$}
                    \State $c \gets c + 1$
                \EndIf
            \EndFor
            \If{$c > \intinf{\frac{b-a}{2}}$}
                \State \Return $l$
            \EndIf
        \EndIf
        \State \Return $-1$
    \EndIf
\EndFunction
\end{algorithmic}
\caption{\label{alg_majority_element}trovare il \emph{majority element} in un array}
\end{algorithm}

\clearpage

\section{Partizione}

\begin{esercizio}
Sia $A[1,\ldots,n]$ un array di interi, con valori da $0$ a $K$ (quindi $0 \le A[i] \le K$ per ogni indice $i$). Dare lo pseudocodice di un algoritmo per trovare una partizione $(X,Y)$ dell'insieme $\{1,2,\ldots,n\}$ tale che $\abs{S_X - S_Y}$ \`e minimo, dove $S_X$ e $S_Y$ sono la somma degli elementi di $X$ e di $Y$ rispettivamente. Per esempio, se $A = [1,3,1,1]$, allora $X = \{1,3,4\}$ e $Y = \{2\}$ \`e la soluzione ottimale con $S_X = 3$, $S_Y = 3$. Se $A = [2,2,3,2]$ allora $X = \{1,2\}$ e $Y = \{3,4\}$ \`e una soluzione ottimale con $S_X = 4$ e $S_Y = 5$.
\end{esercizio}

\clearpage

\end{document}