\section{27 ottobre 2014}

\begin{enumerate}
    \item (2.44)
    \item (2.45)
    \item (2.47)
    \item (2.48)
    \item (3.1)
    \item (3.6)
\end{enumerate}

Lancio 2 dadi. E = ``almeno uno dei due d\`a 6'', F = ``i dadi danno numeri diversi''. Calcolare P(E|F).

Lo spazio campionario $S = \{(a, b) : a, b \in \{1 \dots 6\}\}$ ha esiti equiprobabili, quindi:
\[
P(E | F) = \frac{|E \cap F|}{|F|}
\]
$|F| = 6 \cdot 5 = 30$. Bisogna sapere $|E \cap F|$.

% disegnare il quadratone di lato sei e togliere la diagonale.