\documentclass[11pt,a4paper,twoside,draft]{report}
% draft mode prevents link generation

\usepackage[italian]{babel}
% per date e ToC in italiano
\usepackage{hyperref}
\usepackage{nameref}
\usepackage[table]{xcolor}
\usepackage{amsmath}        % matematica
\usepackage{amsthm}         % teoremi
\usepackage{amssymb}        % simboli
\usepackage{amsfonts}       % font matematicosi
\usepackage{mathrsfs}
\usepackage{mathtools}
\usepackage{thmtools}
\usepackage{centernot}      % per semplificare
\usepackage{pgfplots}       % per i grafici
\pgfplotsset{compat=1.5}    % consigliata compatibilita'
                            % con la versione 1.5

\usepackage{fullpage}       % troppo margine normalmente
\usepackage{parskip}        % preferisco spazio fra i paragrafi
                            % all'indentazione sulla prima riga
\usepackage{fancyhdr}

\theoremstyle{plain}% default
\newtheorem{theorem}{Teorema}[section]
\newtheorem{lem}[theorem]{Lemma}
\newtheorem{prop}[theorem]{Proposizione}
\newtheorem{cor}[theorem]{Corollario}

\theoremstyle{definition}
\newtheorem{defn}{Definizione}[section]
\newtheorem{conj}{Congettura}[section]
% \newtheorem{exmp}{Esempio}[section]
\newtheorem*{esercizio}{Esercizio}

\newenvironment{exmp}
{ \par\medskip\textbf{Esempio:}\newline\rule{\textwidth}{0.4pt}\newline }
{ \newline\rule{\textwidth}{0.4pt}\medskip }


\theoremstyle{remark}
\newtheorem*{comm}{Commento}
\newtheorem*{oss}{Osservazione}
\newtheorem*{note}{Nota}
\newtheorem{caso}{Caso}

\pagestyle{fancyplain}
\fancyhead{} % clear all header fields
\fancyfoot{} % clear all footer fields
\fancyfoot[C]{\today}
\fancyfoot[LE,RO]{\thepage}
\renewcommand{\headrulewidth}{0pt}
\renewcommand{\footrulewidth}{0pt}

\usetikzlibrary{shapes,arrows,calc,fit,backgrounds}

\newcommand{\covers}{<\!\cdot}
\newcommand{\compl}{\mathsf{c}}
\newcommand{\dotcup}{\mathaccent\cdot\cup}
\newcommand{\lateralsx}[1]{{}_{#1}\!\sim}
\newcommand{\lateraldx}[1]{\sim_{#1}}
\newcommand{\mcm}{\text{mcm}}
\newcommand{\mcd}{\text{MCD}}
\newcommand{\pow}[1]{<\!{#1}\!>}
\newcommand{\abs}[1]{\left\lvert{#1}\right\rvert}
\newcommand{\divides}{\bigm|}
\newcommand{\definition}{\overset{\underset{\mathrm{def}}{}}{=}}
\newcommand{\supop}{\vee}
\newcommand{\infop}{\wedge}
\newcommand{\subgroupset}{\mathscr{S}}
\newcommand{\naturals}{\mathbb{N}}
\newcommand{\integers}{\mathbb{Z}}
\newcommand{\rationals}{\mathbb{Q}}
\newcommand{\reals}{\mathbb{R}}
\newcommand{\complexes}{\mathbb{C}}
\newcommand{\field}{\mathbb{K}}
\newcommand{\matrices}{\mathfrak{M}}

\renewcommand{\labelitemi}{$-$}
\let\oldforall\forall
\renewcommand{\forall}{\ \oldforall \ }
\let\oldexists\exists
\renewcommand{\exists}{\ \oldexists \ }
\renewcommand{\implies}{\Rightarrow}
\renewcommand{\iff}{\Leftrightarrow}

\makeatletter
\let\orgdescriptionlabel\descriptionlabel\renewcommand*{\descriptionlabel}[1]{%
\let\orglabel\label
\let\label\@gobble
\phantomsection
\edef\@currentlabel{#1}%
%\edef\@currentlabelname{#1}%
\let\label\orglabel
\orgdescriptionlabel{#1}%
}
\makeatother

\begin{document}

\title{Appunti di Algebra}
\author{Michele Laurenti \\ \href{mailto:asmeikal@me.com}{asmeikal@me.com}}
\date{\today}

\maketitle

\newpage

\tableofcontents

\newpage

\begin{abstract}
I riferimenti a teoremi, proposizioni, propriet\`a e altro sono ipertestuali (ci puoi clickare). Anche altri elementi del pdf lo sono, come i siti internet indicati di seguito, l'indice, o la mia mail in prima pagina.

Il sito della professoressa Venezia \`e: \url{http://twiki.di.uniroma1.it/twiki/view/Algebra/MZ/WebHome}.

Il codice di questi appunti \`e disponibile su \href{https://github.com/asmeikal/Appunti-Algebra-2014}{GitHub} e distribuito con una licenza \href{http://creativecommons.org/licenses/by-nc-sa/4.0/deed.it}{Creative Commons Attribuzione - Non commerciale - Condividi allo stesso modo 4.0 Internazionale}. Sei libero di modificare, elaborare e condividere il codice a scopo non commerciale, indicandomi come autore e distribuendolo con la stessa licenza.

Se ti ho dato una copia di questi appunti intendila come una copia personale, e in quanto tale non diffonderla.
\end{abstract}

\part{Algebra}

\chapter{Nozioni e concetti fondamentali}

\input{concetti_base.tex}

\chapter{Strutture algebriche}

\input{strutture.tex}

\part{Algebra lineare}

\chapter{Spazi vettoriali}

\input{spazi_vettoriali.tex}

\chapter{Risoluzione di sistemi lineari}

\input{risoluzione.tex}

\chapter{Applicazioni lineari}

\input{applicazioni.tex}

\newpage

\renewcommand{\listtheoremname}{Indice delle definizioni}
\listoftheorems[ignoreall,show={defn},numwidth=3em]

\renewcommand{\listtheoremname}{Indice dei teoremi}
\listoftheorems[ignoreall,show={theorem},numwidth=3em]

\renewcommand{\listtheoremname}{Indice delle proposizioni}
\listoftheorems[ignoreall,show={prop},numwidth=3em]


\end{document}